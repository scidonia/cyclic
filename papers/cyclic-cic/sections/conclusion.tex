\section{Conclusion}

This paper proposes and mechanizes a cyclic variant of a CIC-like calculus.
The main contributions are:
\begin{itemize}
\item \textbf{A cyclic CIC calculus.}
  We give (to our knowledge) the first development of a CIC-style setting where proof objects are finite graphs with cycles, rather than trees with a fixed recursion/induction discipline.
\item \textbf{Cyclic normalization as supercompilation.}
  We make a direct connection between cyclic proof normalization and supercompilation: driving steps correspond to local computation/commuting conversions, while fold/refold and back-links correspond to supercompiler-style memoization and generalization, governed by a termination control (homeomorphic embedding).
\item \textbf{Verified meta-theory and transformation correctness.}
  Substantial parts of the pipeline and its supporting theorems are verified in Coq, including key semantic preservation results and the read-off/extraction round-trip theorem.
\end{itemize}

\paragraph{Verified results and current status.}
The mechanized development currently includes:
\begin{itemize}
\item a proved CIU preservation theorem for head beta reduction (Theorem~\ref{thm:beta-ciu});
\item a proved round-trip identity theorem for read-off and extraction (Theorem~\ref{thm:extract-read-off-id}) and its CIU corollary (Theorem~\ref{thm:extract-read-off-ciu});
\item a proved ranking-based sufficient condition for the global progress predicate (Proposition~\ref{prop:ranking-nat});
\item a proved CIU preservation theorem for the \textsc{CaseCase} commuting conversion (Theorem~\ref{thm:casecase-ciu}).
\end{itemize}

Future work is to complete the remaining transformation theorems, strengthen the canonical-form theory, and refine the notion of canonical cyclic representatives (graph equivalence and refolding invariants).
