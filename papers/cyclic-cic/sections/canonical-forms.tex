\section{Canonical forms via supercompilation-style control}
\label{sec:canonical}

The long-term goal is a theory of \emph{canonical cyclic proofs}: intensional representatives stable under semantics-preserving transformations.
In this project, the intended normalization procedure is explicitly inspired by supercompilation \cite{turchin-1986-supercompiler,sorensen-gluck-1995-supercompilation}.

\subsection{Control relation (homeomorphic embedding)}

Supercompilation requires a termination control (a ``whistle'') preventing infinite unfolding.
The repository implements a homeomorphic embedding relation on terms and extends it to typed judgements (\texttt{theories/Progress/Embedding.v}).
We write \(j_1 \preceq j_2\) for judgement embedding.

The classical meta-theorem (by Kruskal-style arguments \cite{kruskal-1960-wqo}) is that homeomorphic embedding is a well-quasi-order on finite terms, which implies that an infinite sequence must contain an embedding pair.
Operationally, this justifies the control rule:

\begin{quote}
If a newly produced configuration embeds a previous one, stop unfolding and refold (generalize/backlink) instead.
\end{quote}

\subsection{Reductions ``modulo unfolding''}

Let \(\to\) be the operational step relation on terms (Section~\ref{sec:source}).
Let \(\rightsquigarrow\) denote a single \emph{unfolding} step at a cyclic back-link (or, in the source calculus, unfolding a \texttt{fix}).
We treat \(\rightsquigarrow\) as admissible but controlled: it is only permitted when it does not violate the control relation.

\subsection{Canonical form (intended)}

We say a cyclic object is in \emph{canonical form} if:
\begin{enumerate}[label=(\arabic*)]
\item (No further reductions) There is no reduction step \(p \to p'\) available in the cyclic object.
\item (Unfolding is controlled) Any unfolding step \(p \rightsquigarrow p_1\) that would enable further reductions \(p_1 \to^{+} p_2\) is forbidden by the control relation: along the would-be unfolding/reduction trace, some configuration embeds a previous one (a ``whistle'' blows), so the procedure must refold rather than continue unfolding.
\end{enumerate}

This definition mirrors supercompilation practice: normal forms are not merely ``stuck''; they are \emph{maximally reduced subject to a termination control}.
The presence of cycles (back-links) turns this into a canonicalization problem for graphs rather than trees: refolding chooses cycle targets and substitutions, and the global progress condition validates that the chosen cycles are productive/sound.

\subsection{Connection to transformations}

Under this view, transformations like head beta and \textsc{CaseCase} are the ``driving'' steps, while read-off/extract and fold/refold manipulate the sharing and cycles.
The control relation (embedding on typed judgements) is the mechanism that prevents infinite unfolding and makes the canonical form finite.
